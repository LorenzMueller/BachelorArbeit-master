%%%%%%%%%%%%%%%%%%%%%%%%%%%%%%%%%%%%%%%%%%%%%%%%%%%%%%%%%%%%%%%%%%%%%%%%%%%%%%%%
% TUM-Vorlage: Wissenschaftliche Arbeit
%%%%%%%%%%%%%%%%%%%%%%%%%%%%%%%%%%%%%%%%%%%%%%%%%%%%%%%%%%%%%%%%%%%%%%%%%%%%%%%%
%
% Rechteinhaber:
%     Technische Universität München
%     https://www.tum.de
% 
% Gestaltung:
%     ediundsepp Gestaltungsgesellschaft, München
%     http://www.ediundsepp.de
% 
% Technische Umsetzung:
%     eWorks GmbH, Frankfurt am Main
%     http://www.eworks.de
%
%%%%%%%%%%%%%%%%%%%%%%%%%%%%%%%%%%%%%%%%%%%%%%%%%%%%%%%%%%%%%%%%%%%%%%%%%%%%%%%%


%%%%%%%%%%%%%%%%%%%%%%%%%%%%%%%%%%%%%%%%%%%%%%%%%%%%%%%%%%%%%%%%%%%%%%%%%%%%%%%%
\input{./Ressourcen/Praeambel.tex} % !!! NICHT ENTFERNEN !!!
%%%%%%%%%%%%%%%%%%%%%%%%%%%%%%%%%%%%%%%%%%%%%%%%%%%%%%%%%%%%%%%%%%%%%%%%%%%%%%%%

\renewcommand{\Thema}{%
    Einteilung von Lerntypen anhand von Blickbewegungen und ihr Einfluss auf das Lösen 
    mathematischer Probleme}

%%%%%%%%%%%%%%%%%%%%%%%%%%%%%%%%%%%%%%%%%%%%%%%%%%%%%%%%%%%%%%%%%%%%%%%%%%%%%%%%
\input{./Ressourcen/Anfang.tex} % !!! NICHT ENTFERNEN !!!
%%%%%%%%%%%%%%%%%%%%%%%%%%%%%%%%%%%%%%%%%%%%%%%%%%%%%%%%%%%%%%%%%%%%%%%%%%%%%%%%

\begin{document}



\title{Einteilung von Lerntypen anhand von Blickbewegungen und ihr Einfluss auf das Lösen 
        mathematischer Probleme}
\author{Lorenz Müller}
\date{16.07.2018}


\section*{Abstract}

\textit{Lernen kann in unterschiedliche Typen unterteilt werden. Eine solche Unterteilung kann zum einen in Sozialformen, wie es bei Schrader passiert ist, als auch in kompetenzbezogene Formen, wie es bei Zech der Fall ist, vorgenommen werden. Zum anderen können Aufgaben anhand ihres Grades an Nützlichkeit, in Form von beigelegten Bildern, haben. Diese Bilder können entweder nur dekorativen Nutzen haben oder auch essenziell für das lösen der Aufgabe sein und somit zusätzlich eine Anschauung vermitteln. In vorliegender Studie wurden 39 Studenten der TU München in einer Eyetrackingstudie zu den Fragestellungen untersucht: Lassen sich die Studenten in Lerntypen anhand ihrer Augenbewegungen einteilen? Wie wirkt sich der Lerntyp auf die Leistungsfähigkeit der Probanden aus? Und gibt es Lerntypen, die mit bestimmten beigelegten Bildern, besser oder schlechter Umgehen können?}

\textit{Um diese Fragen beantworten zu können, wurde den Studenten zu Beginn ein heuristisches Lösungsbeispiel für einen mathematischen Zusammenhang gezeigt. Die Blickbewegungen der Studenten auf diesem Lösungsbeispiel legt die Lerntypunterteilung der Probanden fest. Im weiteren Teil wurden mathematische Aufgaben mit dekorativen und essenziellen Bildern gestellt, welche die Probanden bearbeiten mussten. Eine Unterteilung der Lerntypen war in der Mehrheit der Probanden möglich, jedoch die Unterschiede ihrer Leistungsfähigkeit nicht signifikant. Bei der Verwendung unterschiedlicher Bildtypen wurden Aufgaben mit essenziellen Bildern etwas besser bearbeitet, als Aufgaben mit nur dekorativen Bildern. Bei der Auswertung, wie die unterschiedlichen Lerntypen mit den beigefügten Bildern umgegangen sind, war festzustellen, dass Lernende, die im ersten Teil der Studie sehr viel auf Bilder geschaut haben, im zweiten Teil mit essenziellen Bildern nicht gut umgehen konnten.}


\tableofcontents % Inhaltsverzeichnis

\chapter{Einleitung}

In der Ausbildung von Schülern, Schülerinnen (SuS) und Studenten ist es wichtig, dass die Lernendeneinen Weg finden sich Lernstoff anzueignen und diesen anwenden zu können. Lerntypen können in auditiv, visuell, motorisch, und kommunikativ unterschieden werden. Der Auditive tut sich beim lernen über zuhören am leichtesten, der Visuelle über Veranschaulichungen, der Motorische eignet sich den Stoff über Bewegungenen an und der Kommunikative nimmt den Lerngegenstand am besten auf, wenn er darüber mit anderen spricht.  Somit sind diese Typen sehr unterschiedlich in ihrer Verwendung von Verarbeitungskanälen (Sinnesorganen).
Jeder dieser Typen schafft es also auf verschiedene Art und Weiße sich neuen Stoff effektiv anzueignen.
Um dies zu gewährleisten, wird bereits im Lehrplanplus der Grundschule in Bayern festgehalten:
Es ist " Aufgabe aller Bildungsorte, in allen Lebensphasen und -bereichen individuelles (fett)
Lernen anzuregen und so zu unterstützen, dass es lebenslang selbstverständlich wird."
\cite{LehrplanGrundschule}
Hierbei entwickeln sich im Laufe der Schulbahn eigene Strategien, um komplexe Sachverhalte,
wie zum Beispiel mathematische Beweise zu erlernen. 
Im Umgang des Lernens aus Texten erscheint es sinnvoll, eine andere Lerntypunterteilung anzustreben, da Worte oder Bilder vorerst gelesen oder betrachtet werden müssen, bevor sie verarbeitet werden können. Somit muss auch ein auditiver oder ein kommunikativer Lerntyp vorerst den Text lesen, um den Inhalt für sich zugreifbar zu machen. 

So wie sich die Lernenden in ihren Typen unterscheiden lassen, ist es auch bei den Hilfestellungen zu Aufgaben. Eine Aufgabenstellung kann nicht nur in ihrer kognitiven Prozessdimension (cognitive process Dimension) oder ihrer Wissensdimension (Knowledge Dimension), sondern auch in der Art der Unterstützung eingeteilt werden. Im mathematischen Kontext werden diese Unterstützungen meist in Beispielrechnungen oder Graphiken wiedergegeben. Graphiken können laut Zitation in unterschiedliche Kategorien eingeteilt werden, welche für die Bearbeitung der Aufgabestellung unterschiedlich hilfreich sind\cite{anderson2001taxonomy}. 

In vorliegender Studie wird die visuelle Wahrnehmung genauer untersucht und die Effektivität von unterschiedlichen Lerntypen im lösen von unterschiedlichen mathematischen Aufgabestellungen untersucht. 
\chapter{Theoretischer Hintergrund}

\section{Lerntypeneinteilung nach \citeauthor{gagnebedingung} (\citedate{gagnebedingung})}

Gagné unterteilt seine Formen des Lernens hierarchisch. Somit setzen die nachfolgenden Lernarten die vorhergehenden Lernarten voraus. Generell wird in zwei Teile unterteilt: 


\begin{itemize}
\item (A) Grundformen des Lernens: Assoziationen und Ketten 
    \begin{itemize}
        \item (A.1) Signallernen
        \item (A.2) Reiz-Reaktions-Lernen
        \item (A.3) Kettenbildung
        \item (A.4) Sprachliche Assoziation
    \end{itemize}
\item (B) Intellektuelle Fähigkeiten
    \begin{itemize}
        \item (B.1) Diskriminationslernen
        \item (B.2) Begriffslernen
        \item (B.3) Regellernen 
        \item (B.4) Problemlösen
    \end{itemize}
\end{itemize}

\subsection[]{(A.1) Signallernen (\citeauthor{pawlow1977klassische} `s (\citedate{pawlow1977klassische}) "klassische" Konditionierungslehre)}

Man versucht mit einem Reiz eine dadurch bedingte Reaktion hervorzurufen. Eines der bekanntesten psychologischen Phänomene dazu, ist der \grqq Pawlowsche Hund \grqq. Hierbei wird immer, bevor ein Hund Futter bekommt eine Glocke geschlagen. Nach einigen Durchgängen ertönt nur noch das Klingen der Glocke, was den Speichelfluss des Hundes anreizt, ohne dass er Futter bekommt. Somit können einfache Reize ( z.B Ton), bestimmte Reaktionen (z.B Speichelfluss) auslösen, welche nicht reflexartig angeboren, sondern antrainiert wurden.

\subsection[]{(A.2) Reiz-Reaktions-Lernen (Trial and Error Prinzip, Lernen durch Verstärkung)}

Der Lernende versucht etwas so lange auf verschiedene Art und Weise, bis es klappt und merkt sich anschließend wie es zum Erfolg führte. 
Beispiel: Jemand ist sich bei der letzten Nummer seines Fahrradschlosses unsicher. Er probiert deswegen  alle Möglichkeiten von 0 – 9 aus und merkt sich bei welcher Zahl das Schloss aufgegangen ist. Hierbei erfährt der Lernende eine Verstärkung, weil er danach mehr kann als davor ( in unserem Beispiel: Schloss öffnen) und dies nur durch einfaches Ausprobieren geschafft hat.

\subsection[]{(A.3) Kettenbildung (Lernen von Abläufen)}

In diesem Fall werden längere Reiz-Reaktions-Ketten gebildet. Hierbei sind alle Formen von Algorithmen oder Verfahren eingeschlossen. Beispiele: Kochen, Telefonieren, oder auch Klammern im Mathematikunterricht auflösen. Diese Ketten können meist verstanden und durchgeführt werden, ohne einen tieferen Sinn des einzelnen Verfahrens verstanden zu haben. In der Schule kann diese Lernart zu Problemen führen, da die \gls{SuS} kein Verständnis für die Zusammenhänge haben müssen, lediglich ihre Algorithmen durchgehen können. Dies kann in Aufgaben mit höherem Abstraktionsgrad zu Problemen führen, wenn das Schema nicht mehr ohne Weiteres anwendbar ist.

\subsection[]{(A.4) Sprachliche Assoziation (verbales auswendig Lernen)}

Hierbei werden bestimmte Definitionen, Gedichte oder Ähnliches so lange wiederholt, bis sie auswendig vom Lernenden vorgetragen werden können. Somit findet eine Verkettung oder auch Assoziation von einfachen Objekten - in unserem Fall Wörtern - statt. Es werden ebenso, wie in den anderen Grundformen, lediglich die Verknüpfungen der Wörter geschaffen und kein Verständnis dabei erzeugt. 

\subsection[]{(B.1) Diskriminationslernen (Unterscheidungslernen)}

In diesem Fall soll der Lernende verschiedene Gegenstände und Merkmale als verschieden erfassen können. Beispiele: Ein bestimmtes Passwort passt nur zu einem bestimmten Account. Oder: Nicht jedes Fahrrad sieht gleich aus. Es gibt Rennräder, Stadträder, Lastenräder und viele mehr. Der Lernende soll hierbei ein besseres Verständnis für Begriffe entwickeln, indem er lernt, diese zu differenzieren.

\begin{figure}[!ht]
\noindent\hspace{0.5mm}\includegraphics[width=12cm]{./Ressourcen/Begriffslernen.png}
\caption{Diskriminationslernen, Zech}
\end{figure}

\subsection[]{(B.2) Begriffslernen (Gemeinsamkeiten finden)}

Bei diesem Typ soll genau der konträre Weg zum Diskriminationslernen begangen werden. Es sollen unterschiedliche Objekte, als \textit{gleich} zusammengefasst werden können. 
Beispiele: ein Rechteck und ein Quadrat sind beides Vierecke, oder ein Audi Q7 und ein BMW X5 sind SUVs. Somit können bessere Verbindungen zwischen Lerngegenständen hergestellt und sich auf allgemeine Details zurückgezogen werden (z.B Winkelsumme Viereck sind 360 Grad, Auto hat einen Motor).


\begin{figure}[!ht]
\noindent\hspace{0.5mm}\includegraphics[width=12cm]{./Ressourcen/Diskrimination.png}
\caption{Begriffslernen, Zech}
\end{figure}

\subsection[]{(B.3) Regellernen (Regeln für Anwendungsbereiche lernen)}

Beim Regellernen ist nicht nur das einfache auswendig Lernen von Regeln oder Ähnlichem gemeint, sondern das damit verbundene Verständnis. Somit kann also der Kontext des Satzes des Pythagoras aufgefasst werden: Zum einen ist die Formel $a^2$ + $b^2$ = $c^2$ im aktiven Wissen vorhanden, zum anderen sind auch die Anwendungen und die Bedingungen der Formel verständlich.

\subsection[]{(B.4) Problemlösen (Aufgaben mit eigenen Überlegungen lösen)}

Dies ist der nächste Schritt vom Regellernen, da hierbei Regeln verstanden sein müssen und diese kombiniert in einer Aufgabe angewendet werden können. Somit gehört in der Gagnéschen Hierarchie der Problemlöser zur höchsten Ordnung, da seine Kompetenzen die Kompetenzen der vorhergehenden Typen beinhalten.

\section{Lerntypeneinteilung nach \citeauthor{zech1983grundkurs} (\citedate{zech1983grundkurs})}

Die Lerntypen von Gagné werden in der Einteilung nach Zech aufgegriffen, zusammengefasst und erweitert, um neue Lerntypen zu definieren. Ebenso betrachtet er die Lerntypen hauptsächlich im Kontext der Mathematik und nicht auf alle Lerngegenstände bezogen. Zech unterteilt seine Lerntypen nicht mehr hierarchisch und gibt klare Lernbediengungen für den jeweiligen Lerntypen an. Es werden die folgenden fünf Typen betrachtet:

\subsection[]{Assoziatives Lernen}

In dieser Unterteilung sind die im kognitiven Bereich beschränkten Grundformen des Lernens (Teil A) nach Gagné zusammengefasst. Somit werden bei diesem Typ \grqq kürzere oder längere Reiz-Reaktions-Verbindungen (Automatismen)\grqq  aufgebaut und jene hauptsächlich auswendig gelernt. Dies kann im mathematischen Sinne als das Anwenden von bekannten Regeln auf einfache Aufgabentypen verstanden werden. Außerdem bezieht sich dieser Typ neben der Abstammung von den Grundformen (A) mehr auf texttuelle als auf bildliche Teile von Aufgaben.
Die Lernbedingung dieses Types ist: häufige Wiederholung. 
Anhand dessen entsteht in der Unterteilung des Versuchaufbaus dieser Arbeit die Gruppierung \gls{Gr1}, welche sich dadurch auszeichnet, dass das heuristische Beispiel sehr häufig wiederholt gelesen wird und somit die Fokussierung auf den Text im Vordergrund steht. 


\subsection[]{Diskriminationslernen}

Es wird zu großen Teilen aus dem Typ von B.1 abgeleitet, wie der Name auch schon vermuten lässt. Hierbei wird aber nochmals verdeutlicht, dass das Diskriminationslernen als Voraussetzung zum Begriffslernen stehen muss, da erst Objekte unter einem Begriff zusammengefasst sein müssen, bevor sie als unterschiedlich erkannt werden können.
Lernbedingungen: Unterschiede hervorheben (zum Beispiel mit bunter Kreide), Kontiguität (Angrenzung, Berührung) (aus \cite{DudenKontiguitaet}).
Dieser Lerntyp ist in der gegebenen Studie schwer von den anderen Typen zu unterscheiden, da sich die Aufgabenstellung nicht explizit mit Diskrimination befasst, sondern mit dem Lösen mathematischer Probleme. Somit wird dieser Lerntyp in der Studie nicht weiter behandelt.

\subsection[]{Lernen mathematischer Begriffe}

Dieser Typ baut sich auch auf dem Lerntyp B.2 Begriffslernen auf. Zunächst werden die Begriffe unterteilt in: Eigenschaftsbegriffe, Relationsbegriffe und zusammengesetzte Begriffe. Eigenschaftsbegriffe beschreiben Merkmale oder Eigenschaften eines Objektes. Relationsbegriffe beschreiben eine Relation zwischen verschiedenen Objekten, wie z.B A hat mehr Kanten wie B. Zusammengesetzte Begriffe werden aus einer Kombination von ursprünglichen Begriffen definiert z.B \textit {eine Teilmenge des euklidischen Raums R n heißt kompakt, wenn sie abgeschlossen und beschränkt ist}.\\ Diese Begriffe sollen durch die Zusammensetzung mehrerer bekannter Begriffe verstanden werden. Dieser Lerntyp baut auf B.3 Regellernen auf. Die Begriffe gehen durch Abstraktion aus der Erfahrungswelt hervor.
Lernbedingungen: relevante Merkmale hervorheben, mehrere Beispiele (irrelevante Merkmale variieren), Gegenbeispiele.
Leider sind die gegebenen Lernbedingungen anhand einer Eyetracking-Studie, bei der nichts markiert wird, oder \gls{dPodP} weder Beispiele noch Gegenbeispiele anbringen kann, nicht auszuwerten. Aus diesem Grund wurde dieser Lerntyp in der Studie ebenfalls nicht weiter behandelt. 

\subsection[]{Lösen mathematischer Probleme}

Unter diesem Lerntyp wird ein Verinnerlichen mathematischer Zusammenhänge verstanden. Hierbei wird das Lernen heuristischer Regeln vorausgesetzt. Es müssen zuerst bestimmte allgemeine Strategien (z.B. Widerspruchsbeweis) aufgenommen werden, um anschließend eine Transferleistung zu vollbringen, wie es bei Aufgaben mit hohem Abstraktionsgrad vorausgesetzt wird. 
Lernbedingungen: Problemlösungsfähigkeiten (analysieren, vergleichen, Beziehungen herstellen uvm.) umsetzen und heuristische Regeln einzusetzen.
In Anlehnung an den Typ Lösen mathematischer Probleme wird in meiner Arbeit die Gruppierung \gls{Gr2} definiert, welche sich sehr stark auf die Bilder fokussiert, nachdem sie den Text das erste Mal aufgenommen hat.

\section{Lerntypeneinteilung nach \citeauthor{schrader2008lerntypen} (\citedate{schrader2008lerntypen})}

Wie bereits angedeutet, unterteilt Schrader seine Lerntypen eher in Anlehnung an allgemeine Persönlichkeitstypen: der Theoretiker, der Anwendungsorientierte, der Musterschüler, der Gleichgültige und der Unsichere. In der vorliegenden Arbeit wird nicht mehr auf alle Typen eingegangen, da sich viele in den Gruppierungen von Zech widerspiegeln und die Persönlichkeitstypen in vorliegender Studie nicht ermittelt sind. Somit wird nur auf die Gruppierung der Unsichere eingegangen.

Der \gls{Gr3} sucht Ursachen bei sich und zweifelt an seinen Fähigkeiten. Das Lernen aus Texten fällt diesem Typ schwer, da er dabei wenig systematisch vorgeht. Er lässt sich lieber etwas mehr Zeit, um die gegebene Aufgabenstellung zu bearbeiten.
Lerncharakteristik: Da ihm das Arbeiten mit Texten Schwierigkeiten bereitet, ist die Bearbeitungszeit länger als gewöhnlich und es werden sowohl Text-, als auch Bildbereiche häufiger als bei gewöhnlichen \gls{PuP} betrachtet. 

\section{Effekte graphischer Darstellungen}

Laut einer Studie von \citeauthor{mayer2005reliability} (\citedate{mayer2005reliability}) hat das Arbeitsgedächtnis zwei Kanäle zur Verarbeitung von Informationen. Einer der Kanäle verarbeitet Informationen von Wörtern, der andere von Bildern. Somit kann eine optimale Lernbedingung geschaffen werden, indem sowohl verständlicher Text als auch passend eingebundene Bilder dem Lernenden zu Verfügung gestellt werden. Laut \citeauthor{bockmannvalue} (\citedate{bockmannvalue}) lassen sich Bilder in drei unterschiedliche Typen unterteilen:
    
    \begin{itemize}
        \item Dekorative Bilder (decorative pictures), welche keine echten Informationen über das gegebene Problem zeigen. Z.B. eine Radfahrerin, wenn es in der Aufgabe über das Zurücklegen einer Strecke mit dem Rad geht.
        \item Repräsentative Bilder (representational pictures), welche Teile der gestellten Aufgabe veranschaulichen. Z.B. ein Drachenviereck, wenn die Aufgabe gestellt ist: den Flächeninhalt eines Drachenvierecks zu bestimmen. 
        \item Essenzielle Bilder (essential pictures), welche Informationen geben die für die Bearbeitung der Aufgabe essenziell sind. Z.B. bei der Berechnung einer Fläche eine Graphik, in der relevante Längen von Seiten eingetragen sind.
    \end{itemize}

In der Studie wurden 217 \gls{SuS} aus der 9. Jahrgangsstufe in Gruppen eingeteilt. Bei dem Versuch wurden die \gls{SuS} in drei Gruppen unterteilt, welche dieselbe Aufgabenstellung erhielten, jedoch die Bildtypen sich unterschieden. In der Aufgabe wurde von den \gls{SuS} verlangt, den Abstand von einer Person zu ihrem Lenkdrachen mit Hilfe des Satzes des Pythagoras zu berechnen. Die Aufgabestellung sollte von den \gls{SuS} nicht bearbeitet, sondern lediglich Fragen hierzu beantwortet werden. 


Die erste Gruppe erhielt die Aufgabenstellung mit einem dekorativen Bild (ein einfaches Abbildung eines Lenkdrachens).


Die zweite Gruppe erhielt ein Bild, in dem die beschriebene Situation (Person mit Schnurverbindung zum Lenkdrachen) bildlich dargestellt wurde (repräsentativ).


Die dritte Gruppe hingegen erhielt dasselbe Bild wie die zweite Gruppe, zusätzlich mit Längenangaben (essenziell).


Es konnte gezeigt werden, dass dekorative Bilder den \gls{PuP} kaum eine Unterstützung in der Lösung der gestellten Aufgaben geben. Hingegen repräsentative einen positiven Effekt und essenzielle Bilder den besten Effekt bewirkten.


\chapter{Methodik}

\section{Eyetracking}

In dieser Studie wurde die Methode des Eyetrackings benutzt um die Blickbewegungen der Probanden auf dem Bildschirm fest zu halten. Im folgenden wird kurz die Einrichtung und die Funktionsweise eines Eyetracking Monitors beschrieben. 

Geräte, welche Blickbewegungen aufzeichnen können unterscheiden sich in zwei grundlegende Kathegorien:
    \begin{itemize}
        \item die eine Aufzeichnung über eine Brille erstellen
        \item die eine Kamera unterhalb des Bildschirms verwenden
    \end{itemize}


In diesem Versuchsaufbau wurde die 2. Variante gewählt. Hierbei ist es sehr wichtig, dass der Proband auf der richtigen Höhe sitzt und einen ungehinderten Blick (keine Brille) auf den Bildschirm hat. Ebenso wichtig ist, dass der Proband seinen Kopf gerade hält und nur die Augen bewegt, bei einer übermäßigen Bewegung des Kopfes werden die aufgezeichneten Daten verfälscht. Bevor mit der eigentlichen Studie begonnen werden kann, muss die Kamera kalibriert werden, hierfür soll der Versuchsteilnehmer einem Punkt am Bildschirm folgen um eine richtige Einstellung des Gerätes hervorzurufen. 

Der Versuchsaufbau ist wie in der Abbildung Versuchsaufbau beschrieben, dass der Proband (P) vor dem Bildschirm mit Kamera sitzt und der Versuchsleiter (V) auf der anderen Seite des Bildschirms. Die Kamera, welche die Blickbewegungen des Probanden aufzeichnet sitzt unter dem Anzeigebildschirm. Dieser Aufbau wurde gewählt, damit der Proband einfach ``weiter'' oder das Ergebnis einer Aufgabe sagen kann ohne den Blick vom Anzeigebildschirm zu lösen. Bei einer der Verwendung einer Tastatur, würde der Proband jedes mal wenn er auf sie schaut, die Kalibrierung des Eyetrackinggerätes verändert und somit die Daten verfälschen. 

\begin{figure}[!ht]
\noindent\hspace{0.5mm}\includegraphics[width=8cm]{./Ressourcen/Versuchsaufbau.png}
\caption{Versuchsaufbau}
\end{figure}

\section{Begriffserklärung und Instrumentenbeschreibung}
Um diese Studie nachzuvollziehen werden im folgenden Teil einige Begriffe erklärt und angegeben wie die Daten ausgewertet wurden. 

Beginnend mit dem Begriff Fixationen auf ein bestimmtes Objekt ist in dieser Studie ein messbares innehalten des Probanden auf einer bestimmten Stelle des Bildschirms gemeint.``Diese Zeitdauer beträgt typischerweise 100 bis 200ms\cite{EyetrackingFixation} .''

Somit sind die Fixationen bei einer Auswertung von Eyetrackinguntersuchungen immer in drei Teile unterteilt:
    \begin{itemize}
        \item Die Koordinaten der einzelnen Fixation, meist aufgeteilt in einen x und einen y Wert.(Horizontal, Vertikal)
        \item Die Dauer der jeweiligen Fixation
        \item Die zeitliche Reihenfolge der Fixation
    \end{itemize}


Durch AOI`s (``Area of Interrest'') kann der für den Probanden sichtbare Bereich des Bildschirms in Abschnitte unterteilt werden, welche für die Auswertung relevant sind. In vorliegender Studie wurden auf der ersten Seite AOIs über die Textabschnitte, sowie über die Graphiken legt um die Fixationen auf diesem Bereich feststellen zu können. Ebenso wurde in den letzten Bereich eine AOI gelegt um das Unterscheiden von dem ersten zum zweiten Durchgang zu ermitteln (mehr dazu im Teil eigene Einteilung).

Eine Heatmap ist eine Darstellung des Bildschirms, in der die Benutzerfixationen anhand von ihrer Häufigkeit und Dauer eingefärbt werden. Hierbei können Zonen, welche vom Benutzer intensiv Betrachtet werden schnell veranschaulicht werden.

Diese Daten wurden mit Hilfe des Eyetrackingapperates RED 500 festgehalten und konnten in unter Verwendung des Programms IViewx unterteilt werden. Somit konnte die Anzahl der Fixationen in einem Bereich ausgewertet werden. Diese Daten wurde verwendet um die Gruppeneinteilung im folgenden Verlauf der Arbeit vorzunehmen.

Nach der Einteilung der Gruppen wurden zum einen Varianzanalysen zu den gegebenen Forschungsfragen angestellt. Die Auswertung erfolgte über einen einfaktorielle ANOVA um signifikante Unterschiede in den Gruppen oder unter den Bildern feststellen zu können. Zum anderen wurden die Mittelwerte berechnet und verglichen. Diese Auswertungen sind im Teil Ergebnis festgehalten.

\section{Ablauf der Studie}

Auf der ersten Folie wurde den Versuchsteilnehmern ein Beweis von Teilbarkeit  von 3 oder 5 aufeinander folgender Zahlen anhand eines Heuristischem Lösungsbeispiels veranschaulicht. Hierbei wurde gezeigt, dass 3 aufeinander folgende Zahlen z.B 3,4,5 in der Summe, also in unserem Beispiel 12, wieder durch die Anzahl der aufeinander folgenden Zahlen teilbar mit Ganzzahligem Ergebnis ist. Somit kann in unserem Beispiel die 12 durch 3 geteilt werden, ohne einen Rest zu haben.
Analog funktioniert es bei jeder ungeraden Anzahl an aufeinander folgender Natürlicher Zahlen. Die erste Folie unterteilt sich in Heuristiken, welche den allgemeinen Lösungsweg des Beweises aufzeigen soll und in Inhalt, welcher von textueller und bildlicher Form sein kann. In bildlicher Form ist zum einen ein Anfangsbeispiel, bei dem die Hypothese bei unterschiedlichen Zahlen überprüft wird. Zum anderen eine Möglichkeit, wie man durch Sortierung den Sachverhalt besser zu veranschaulichen. Im textueller Form wurden die Überlegungen festgehalten.

Darauf folgend wurden den Probanden 9 mathematische Aufgaben gestellt, welche sie mit Hilfe einer multiple choice Antwortmöglichkeit beantworten sollten. Ein Beispiel für eine Aufgabe ist:

\begin{figure}[H]
\noindent\hspace{0.5mm}\includegraphics[width=17cm]{./Ressourcen/Radfahrerin.png}
\caption{Radfahrerin}
\end{figure}

\section{Eigene Aufgabe-Bildeinteilung}

In dieser Studie wurden die zugegebenen Bilder und Graphiken wie folgt unterteilt:

In ersten beiden Aufgabestellungen geht es um eine Radfahrerin, wobei lediglich ein Bild dieser Radfahrerin abgebildet ist. Dieses Bild trägt nicht zur Veranschaulichung oder Unterstützung der Versuchsperson bei und ist somit dekorativ. 


In der nächsten Aufgabenstellung ist ein Graph über die Geschwindigkeit eines Rennfahrers in Abhängigkeit vom Weg zu sehen. Dieser Graph muss verwendet werden, um die Aufgabenstellung zu bearbeiten und ist somit essenziell. 


In der darauf folgenden geschilderten Problematik sind 2 Graphiken, welche Teile eines Zufallsexperimentes darstellen verwendet. Hierbei müssen die Felder des Glücksrades und die Verteilung der Kugeln betrachtet werden um die Aufgabenstellung lösen zu können und somit sind die Graphiken auch essenziell.


Die nächsten beiden Aufgaben handeln vom Bergsteigen, hierbei ist lediglich ein Berg als Bild gegeben, welcher aber für die Bearbeitung der Aufgabe aber nicht nützlich ist, somit ist das Bild dekorativ. 


Im Anschluss wird noch eine Aufgabe im Themenbereich des Autofahrens gegeben. Dabei wurde wiederum ein Graph verwendet, der die Geschwindigkeit und die Zeit der Autofahrerin in Relation setzt. Diese Graphik ist wiederum essenziell für die Bearbeitung der Aufgabe. 


In den letzten beiden Aufgabestellungen wurde sowohl eine Tabelle, welche essenziell für die Bearbeitung ist, als auch ein einfaches Bild von einem Auto, welches eher dekorativen Charakter hat, verwendet. Da hierbei eine Mischung von Bildtypen verwendet wurde werden diese beiden Aufgaben in der weiteren Bearbeitung nicht weiter betrachtet. 

Somit kommen in dieser Bildunterteilung lediglich drei essenzielle und vier dekorative Bilder vor und keine repräsentativen Bilder. Somit sollte der Unterschied zwischen hilfreichen und nicht hilfreichen Bildern besser sichtbar sein.


\section{Eigene Gruppeneinteilung}

Aus den oben genannten Lerntypen wurden drei Gruppierungen gebildet: \gls{Gr1} der Unsichere, \gls{Gr2} der Problemlöser, \gls{Gr3} der Textuelle. Die gegebenen 39 Probanden wurden in die vorliegenden Typen unterteilt nach folgender Tabelle. 22 der Versuchsteilnehmer konnten in jeweils eine der Gruppierungen unterteilt werden. Bei den restlichen 17 war dies leider nicht möglich. Das erstmalige lesen des Textes dauerte dauerte im Schnitt über alle Probanden 110,4 Sekunden. Diese Dauer konnte ermittelt werden, indem in einer AOI am Ende des Textes Fixationen festgestellt wurden. Sobald also ein Proband in dieses Feld geblickt, ist der erste Durchgang des Lesevorgangs beendet und der Zweite beginnt, welcher andauert, bis der Proband auf die nächste Folie wechselt. Diese Einteilung konnte anhand von Messungen der Zeit und der Reihenfolge der Fixationen belegt werden. 


Die Probanden wiesen in der Allgemeinheit noch folgenden Werte im Durchschnitt auf:
    \begin{itemize}
        \item Dauer des Zweiten Durchgangs: 72,6 Sekungen 
        \item Fixationen auf die Bildbereiche im zweiten Durchgang: 75,2
        \item Fixationen auf die Textbereiche im zweiten Durchgang: 71,1
    \end{itemize}

Die Probanden wurden auf die genannten Merkmale untersucht und in die passen den Gruppen anhand dieser Einteilung zugeordnet.
\section*{Einteilungsschema}

\begin{table}[!h]
\hspace{-5pt}
\begin{tabularx}{\textwidth + 5pt}{| @{\hspace{3pt}} M || @{\hspace{3pt}} M  | @{\hspace{3pt}} M  | @{\hspace{3pt}} M |}
\hline
\textbf{Ausprägungen} & \textbf{Typ Unsicher (Gr1.)} & \textbf{Typ Problemlöser (Gr2.)} & \textbf{Typ Textuell (Gr3.)}\\
\hline
\hline
Dauer 1. Durchgang          & (+) & 0 & 0\\
\hline
Dauer 2. Durchgang          & ++ & 0 & (+)\\
\hline
Textfixationen 2. Durchgang & + & 0 & +\\
\hline
Bildfixationen 2. Durchgang & + & + & 0\\
\hline
\end{tabularx}
\caption{Ausprägungen}
\end{table}



\section{Schwierigkeiten bei der Erfassung}

Bei der Erfassung der Punkte einer Aufgabe ist lediglich ein Punkt vergeben worden, wenn die Aufgabe richtig gelöst wurde. Kein Punkt ist vergeben worden, wenn die Lösung nicht stimmt, somit wurden keine Teilpunkte vergeben. Wenn ein Proband die Aufgabe zu Teilen richtig löst, anschließend aber sich einen Denkfehler erlaubt bekommt er dennoch 0 Punkte auf die Aufgabe. Hierbei findet somit eine ergebnisorientierte- und keine lösungswegorientierte Auswertung der Aufgaben statt, welche die Fähigkeiten der Probanden schlechter abbildet als sie eigentlich wären.

Zusätzlich wurden Multiplechoicefrage Antwortmöglichkeiten gegeben. Diese beiden Voraussetzungen erzeugen eine Unschärfe der Antwort des Probanden. Eine Verbesserung der Leistung des Probanden wird durch die vierfache Antwortmöglichkeit der Aufgaben gegeben, da der Proband durch raten der Richtigen Lösung mit einer Wahrscheinlichkeit von 1/4 oder 25 Prozent richtig liegt. Dies hat zur Folge, dass die Fähigkeiten der Probanden besser abgebildet werden als sie eigentlich wären.  

Ebenso war es für die Probanden nicht möglich sich Aufgabenteile zu markieren, oder sich Notizen oder Rechnungen auf Papier nebenbei zu machen, dies hätte Kalibrierungsprobleme des Eye Tracking Automaten hervorrufen können. Dies könnte man meinen, sollte sich auch negativ aus die Auswertung der Aufgaben auswirken, jedoch haben Schukajlow und Leiss (2011) gezeigt, dass ``(b)ezüglich der selbstberichteten Strategienutzung und der mathematischen Modellierungsleistungen der Lernenden konnten keine signifikanten Korrelationen festgestellt werden\cite{schukajlow2011selbstberichtete}.'' Somit dieser Punkt nicht mit Sicherheit als ein Nachteil des Probanden gewertet werden. 

Die Einschränkungen der Ergebnisauswahl und der nur ergebnisorientierten Auswertungsansatz wurden so gewählt, weil sie die Auswertung der Studie deutlich vereinfacht haben.  Zusätzlich könnten sich auch beide Effekte im besten Fall aufgehoben haben.  

\section{Forschungsfrage}

Texte werden von den Versuchsteilnehmern unterschiedlich bearbeitet, hierbei liegen unterschiedliche Teilbereiche in textueller oder in bildlicher Form im Fokus. Häufig ist das lesen beim ersten Mal ein relativ linearer Vorgang, von links nach rechts und dann von oben nach unten gelesen. Nach dem ersten Durchgang unterscheiden Sich die Probanden und einzelne Teilbereiche des Textes werden einer weiteren Betrachtung unterzogen. Dies führt uns zu der Frage:
(1) Kann man die Probanden anhand ihrer Augenbewegungen in Gruppen unterteilen, (2) ob die Unterteilung Aufschlüsse die Leistungsfähigkeit innerhalb der Gruppe gibt?

Ein weitere Teil der Studie ist, das Ergebnis von Matthias Böckmann und Stanislaw Schukajlow (2018) in unserer Studie zu verifizieren (3), indem die unterteilten Aufgaben und die Gesamtpunktzahl der jeweiligen Aufgabe betrachtet wird. Somit wird untersucht ob ein direkter Zusammenhang zwischen der Lösung der Aufgabe und dem Typ des verwendeten Bildes besteht. Zuletzt wird noch untersucht, ob bestimmte Aufgabentypen für einzelne Probandengruppen positive oder auch negative Auswirkungen haben (4).
\chapter{Ergebniss}

Die in der Stuide getesteten Probanden waren Studenten der TU München und wiesen die folgenden Werte auf:

    \begin{itemize}
        \item 35,9 Prozent von ihnen waren weiblich 
        \item die Probanden kamen überwiegend aus Bayern 66,7 Prozent
        \item das Druchschnittsalter lag bei 21,2 Jahren 
        \item waren im Mittel im 3,85. Fachsemester
    \end{itemize}

 Um die Vorschungsfrage (1) zu untersuchen wurden Gruppen mit Hilfe einer Varianzanalyse auf die Werte: Fixationen auf dem Bild, Fixationen auf dem Text und Dauer nach dem ersten Durchgangs, sowie der Dauer des ersten und des zweiten Durchgangs untersucht.


\begin{figure}[!ht]
\noindent\hspace{0.5mm}\includegraphics[width=15cm]{./Ressourcen/Gruppenunterscheidung.png}
\caption{Gruppenvarianz, Lorenz Müller}
\end{figure}

Wenn man sich der zweiten Vorschungsfrage widment. Kommt man zu folgenden Ergebnissen:
Alle Studenten erziehlten im Schnitt 6,03 Punkte. 
Die jeweiligen Gruppen erziehlten insgesamt über alle Aufgaben im Durchschnitt folgende Punkte:

\begin{table}[!h]
\hspace{-5pt}
\begin{tabularx}{\textwidth + 5pt}{| @{\hspace{3pt}} M || @{\hspace{3pt}} M  | @{\hspace{3pt}} M | @{\hspace{3pt}} M |}
\hline
\textbf{ } & \textbf{Typ Unsicher} & \textbf{Typ Problemlöser} & \textbf{Typ Textuell}\\
\hline
\hline
Punkte        & 6,17 & 5,5 & 6,3\\
\hline
\end{tabularx}
\caption{Mittelwert der Punkte}
\end{table}


Bei der Untesuchung nach signifikanten Unterschieden wurde
die Varianzanalyse über die Gesamtpunktzahl angestellt, welche folgende Ergebnisse hervorbringt:

\begin{figure}[!ht]
\noindent\hspace{0.5mm}\includegraphics[width=15cm]{./Ressourcen/Punktevarianz.png}
\caption{Punktevarianz, Lorenz Müller}
\end{figure}

Wendet man sich nun der dritten Forschungsfrage, des Einflusses der gewählten Bilder zu, ergibt sich.
Betrachtet man die Beantwortung der einzelnen Aufgaben von allen Probanden.

\begin{table}[!h]
\hspace{-5pt}
\begin{tabularx}{\textwidth + 5pt}{| @{\hspace{3pt}} M || @{\hspace{3pt}} M  | @{\hspace{3pt}} M | @{\hspace{3pt}} M |}
\hline
\textbf{Aufgabe} & \textbf{Radfahrerin 1} & \textbf{Radfahrerin 2} & \textbf{Rennfahrer} \\
\hline
\hline
Prozent richtig beantwortet       & 97,5 & 17,9 & 76,9 \\
\hline
\end{tabularx}
\caption{Mittelwert der Punkte}
\end{table}

\begin{table}[!h]
\hspace{-5pt}
\begin{tabularx}{\textwidth + 5pt}{| @{\hspace{3pt}} M | @{\hspace{3pt}} M  | @{\hspace{3pt}} M | @{\hspace{3pt}} M |}
\hline
\textbf{Zufall} & \textbf{Berg 1} & \textbf{Berg 2} & \textbf{Autofahren}\\
\hline
\hline
    51,3 & 76,9 & 74,3 &  94,9\\
\hline
\end{tabularx}
\caption{Mittelwert der Punkte}
\end{table}

Im Schnitt wurden somit die jeweiligen Bildtypen zu folgenden Prozentwerten richtig beantwortet. 

\begin{table}[!h]
\hspace{-5pt}
\begin{tabularx}{\textwidth + 5pt}{| @{\hspace{3pt}} M || @{\hspace{3pt}} M  | @{\hspace{3pt}} M | @{\hspace{3pt}} M |}
\hline
\textbf{Aufgabentyp} & \textbf{dekorativ} & \textbf{essenziell} \\
\hline
\hline
Prozent richtig beantwortet       & 66,7 & 74,3 \\
\hline
\end{tabularx}
\caption{Mittelwert der Punkte}
\end{table}

Bei der Untersuchung nach signifikanten Unterschieden ergibt sich bei der Varianzanalyse:


\begin{figure}[!ht]
\noindent\hspace{0.5mm}\includegraphics[width=15cm]{./Ressourcen/Aufgabenuntescheidung.png}
\caption{Punktevarianz, Lorenz Müller}
\end{figure}




Wiedmet man sich nun der vierten Vorschungsfrage:
Werden bei der Auswetung der einzelnen Gruppen auf die Fragen Ergebnisse fett markiert, welche sich um mehr als 20 Prozent vom Wert der Allgemeinheit unterscheiden:

\begin{table}[!h]
\hspace{-5pt}
\begin{tabularx}{\textwidth + 5pt}{| @{\hspace{3pt}} M || @{\hspace{3pt}} M  | @{\hspace{3pt}} M | @{\hspace{3pt}} M |}
\hline
\textbf{Aufgabe} & \textbf{Radfahrerin 1} & \textbf{Radfahrerin 2} & \textbf{Rennfahrer} \\
\hline
\hline
Prozent richtig beantwortet       & 100 & 10 & 90 \\
\hline
\end{tabularx}
\caption{Typ Problemlöser bei den unteschiedlichen Aufgabenstellungen 1}
\end{table}


\begin{table}[!h]
\hspace{-5pt}
\begin{tabularx}{\textwidth + 5pt}{| @{\hspace{3pt}} M | @{\hspace{3pt}} M  | @{\hspace{3pt}} M | @{\hspace{3pt}} M |}
\hline
\textbf{Zufall} & \textbf{Berg 1} & \textbf{Berg 2} & \textbf{Autofahren}\\
\hline
\hline
    40 & 60 & 70 &  \textbf{50}\\
\hline
\end{tabularx}
\caption{Typ Problemlöser bei den unteschiedlichen Aufgabenstellungen 2}
\end{table}

\begin{table}[!h]
\hspace{-5pt}
\begin{tabularx}{\textwidth + 5pt}{| @{\hspace{3pt}} M || @{\hspace{3pt}} M  | @{\hspace{3pt}} M | @{\hspace{3pt}} M |}
\hline
\textbf{Aufgabe} & \textbf{Radfahrerin 1} & \textbf{Radfahrerin 2} & \textbf{Rennfahrer} \\
\hline
\hline
Prozent richtig beantwortet       & 83 & 17 & 67 \\
\hline
\end{tabularx}
\caption{Typ Unsicher bei den unteschiedlichen Aufgabenstellungen 1}
\end{table}

\begin{table}[!h]
\hspace{-5pt}
\begin{tabularx}{\textwidth + 5pt}{| @{\hspace{3pt}} M | @{\hspace{3pt}} M  | @{\hspace{3pt}} M | @{\hspace{3pt}} M |}
\hline
\textbf{Zufall} & \textbf{Berg 1} & \textbf{Berg 2} & \textbf{Autofahren}\\
\hline
\hline
    \textbf{83} & 83 & 67 &  100\\
\hline
\end{tabularx}
\caption{Typ Unsicher bei den unteschiedlichen Aufgabenstellungen 2}
\end{table}

\begin{table}[!h]
\hspace{-5pt}
\begin{tabularx}{\textwidth + 5pt}{| @{\hspace{3pt}} M || @{\hspace{3pt}} M  | @{\hspace{3pt}} M | @{\hspace{3pt}} M |}
\hline
\textbf{Aufgabe} & \textbf{Radfahrerin 1} & \textbf{Radfahrerin 2} & \textbf{Rennfahrer} \\
\hline
\hline
Prozent richtig beantwortet       & 100 & 17 & \textbf{100} \\
\hline
\end{tabularx}
\caption{Typ Textuell bei den unteschiedlichen Aufgabenstellungen 1}
\end{table}


\begin{table}[!h]
\hspace{-5pt}
\begin{tabularx}{\textwidth + 5pt}{| @{\hspace{3pt}} M | @{\hspace{3pt}} M  | @{\hspace{3pt}} M | @{\hspace{3pt}} M |}
\hline
\textbf{Zufall} & \textbf{Berg 1} & \textbf{Berg 2} & \textbf{Autofahren}\\
\hline
\hline
    67 & 67 & 67 &  83\\
\hline
\end{tabularx}
\caption{Typ Textuell bei den unteschiedlichen Aufgabenstellungen 2}
\end{table}

Somit sind folgende Ergebnisse Auffällig. Wobei mit Hilfe der Varianzanalyse untersuch wurde, ob diese sich signifikant von der Allgemeinheit unterscheiden. 

\begin{itemize}
    \item Autofahrt wurde von dem Typ Problemlöser mit nur 50 Prozent richtig gelöst. (Allgemeinheit 94,9 Prozent) ($\alpha$ = 0,243 F = 1,407)
    \item Zufall wurde von dem Typ Unsicher mit 83,3 Prozent richtig gelöst. (Allgemeinheit 0,513) ($\alpha$ = 0,152 F = 2,135)
    \item Rennfahrer wurde von Typ Textuell von allen Probanden richtig gelöst. (Allgemeinheit 0,769) ($\alpha$ = 0,092 F = 2,990)
\end{itemize}

Betrachtet man noch abschliesend, wie sich die unterschiedlichen Lerntypen mit den Bildtypen verhalten haben, so ergibt sich folgende Tabelle.

\begin{table}[!h]
\hspace{-5pt}
\begin{tabularx}{\textwidth + 5pt}{| @{\hspace{3pt}} M | @{\hspace{3pt}} M  | @{\hspace{3pt}} M | @{\hspace{3pt}} M |}
\hline
\textbf{Lerntyp} & \textbf{Unsicher} & \textbf{Problemlöser} & \textbf{Textuell}\\
\hline
\hline
    dekorativ & 62,5 & 60 &  62,5\\
\hline
    essenziell & 83,3 & 60,0 &  83,3\\
\hline
\end{tabularx}
\caption{Typ Textuell bei den unteschiedlichen Aufgabenstellungen 2}
\end{table}

Mit der Varianzanalyse: 

\begin{figure}[!ht]
\noindent\hspace{0.5mm}\includegraphics[width=15cm]{./Ressourcen/DekoEssGruppen.png}
\caption{Punktevarianz, Lorenz Müller}
\end{figure}



\chapter{Diskussion}

Nach der Betrachtung aller Einzelergebnisse können Rückschlüsse auf die Forschungsfragen vollzogen werden.  

Wie aus der Graphik der Gruppenvarianz entnommen werden kann, unterscheiden sich die Gruppen untereinander in folgenden Merkmalen: Fixationen auf den Text bzw. die Bilder und die Dauer des jeweiligen zweiten Durchgangs.

In der Dauer des ersten Durchgangs ist kein signifikanter Unterschied zwischen den Gruppen festzustellen ($\alpha$ = 0,620 F = 0,490). Dieses Ergebnis stimmt mit der Theorie überein, dass die \gls{PuP} den Text zunächst linear lesen und sich erst danach die Unterschiede ergeben. In der Anzahl der Fixationen auf den Text sowie das Bild im zweiten Durchgang ist ein signifikanter Unterschied festzustellen ($\alpha$ = 0,000 F = 11,991 für das Bild und $\alpha$ = 0,001 F = 10,034 für den Text). Ebenso ist in der Dauer des zweiten Durchgangs ein signifikanter Unterschied zu erkennen ($\alpha$ = 0,000 F = 17,073). Bemnach ist die Forschungsfrage (1) mit ja zu beantworten; man kann die \gls{PuP} in Gruppen unterscheiden.

Verwunderlich ist bei der Auswertung der Mittelwerte der Punkte (Forschungsfrage 2), dass die Problemlöser mit 5,5 Punkten im Schnitt am schlechtesten unter den Gruppen abgeschlossen haben. Dieses Ergebnis ist dahingehend verwunderlich, da in der Lerntypeinteilung von Gangé die Problemlöser, welche sich viel mit den unterschiedlichen Darstellungsformen auseinander setzen, hierachisch über den Textuellen, welche sich aus den Grundformen des Lernes nach Gangé zusammen setzen sollten, stehen. \\
Auf der anderen Seite hat Zech seine Lerntypeinteilung nicht mehr hierarchisch vollzogen, was die Ergebnisse weniger verwunderlich erscheinen lässt. Typ Textuell hat in dieser Auswertung mit 6,3 Punkten im Mittel am besten abgeschlossen und der Typ Unsicher liegt mit 6,17 Punkten im Mittelfeld. Die Punktevarianzabbildung zeigt aber auch, dass die Unterschiede bei den Punkten der einzelnen Gruppen nicht signifikant sind ($\alpha$ = 0,551 F = 0,615). 

Das folgende Zitat von Hyrskykari, Ovaska, Majaranta, Räihä und Lehtinen (2008) erklärt anschaulich, wie leicht es ist, Fehlbeurteilungen bei der Untersuchung von Eyetrackingdaten anzustellen:

``Eine markante Zone auf einer Heatmap wird oft als eine interressante Zone interpretiert. Es hat die Aufmerksamkeit des Benutzers auf sich gezogen und deswegen wird angenommen, dass die Zone für den Benutzer jetzt verständlich ist. Dennoch kann aber auch das Gegenteil der Fall sein: Die Zone hat die Aufmerksamkeit des Benutzers angeregt, weil sie verwirrend und problematisch ist und der Benutzer die dargestellte Information nicht verstanden hat''\cite{hyrskykari2008gaze}.

Ebenso haben Hayhoe und Ballard (2005) in ihrer Studie gezeigt, dass in manchen Situationen Benutzer direkt auf ein relevantes und wichtiges Objekt schauen und dennoch keine Gedächtnisspur (Wissensgewinn) regestrierbar ist\cite{hayhoe2005eye}. 

Daraus lässt sich schleißen, dass es nicht so einfach ist, die Verhaltensmuster (in unserer Studie Blickbewegungen) der Probanden direkt in ihre kognitiven Prozesse zu übersetzen. Somit ist es leicht, sich von den Eyetrackingdaten blenden zu lassen und diese falsch zu interpretieren. In dieser Studie könnten die \gls{PuP} aus der Gruppe Problemlöser lediglich viele Fixationen auf den Bildern haben, weil sie diese nicht ganz verstanden hatten. 

Ebenso ist die Gruppengröße von 6 \gls{PuP} in Gruppe Unsicher, 10 Versuchsteilnehmer in Gruppe Problemlöser und 6 \gls{PuP} in Gruppe Textuell deutlich zu gering, um signifikante Ergebnisse zu generieren. Man könnte auch feststellen, diese Gruppengröße bietet zu wenig statistische Aussagekraft für eindeutige Ergebnisse. 

Widmet man sich der dritten Forschungsfrage:\\
Inwiefern haben die Bilder einen Einfluss auf das Ergebnis der \gls{PuP}, lässt sich schon in den ersten beiden Aufgaben einen Wiederspruch feststellen: \gls{Rad} 1 und \gls{Rad} 2 verwenden beide ein dekoratives Bild und ihre Beantwortung fällt im ersten Aufgabenteil sehr gut mit 97,5 Prozent aus und im zweiten Teil, bei dem dasselbe dekorative Bild verwendet wurde sehr schlecht mit 17,9 Prozent aus. Der deutliche Unterschied in der Beantwortung legt den Schluss nahe, das sich das Niveau der Aufgabenstellung unterscheidet.\\
Die Aufgaben \gls{Ber} wurden zu 76,9 und zu 74,3 Prozent richtig bearbeitet, obwohl das beigelegte Bild ebenso nur dekorativ war. In den restlichen Fällen war die verwendete Graphik jeweils essenziell und wurde mit \gls{Aut} 94,9 Prozent, \gls{Zuf} 51,3 Prozent und \gls{Ren} 76,9 Prozent richtig beantwortet. Hierbei wird die Bearbeitung des Aufgabenteils \gls{Zuf} auffällig, da sie am schlechtesten von den Aufgabenstellungen mit essenziellen Graphiken bearbeitet wurde. 


Im Durchschnitt aller \gls{PuP} wurden die Aufgaben mit dekorativen Bildern zu 66,6 Prozent richtig beantwortet und die Aufgabe mit essenziellen Bildern zu 74,4 Prozent. Dies ist zwar kein signifikanter Untersied ($\alpha$ = 0,749 F = 0,115), zeigt aber eine Tendenz in Richtung des Effekts, der aus der Studie von Matthias Böckmann und Stanislaw Schukajlow (2018) erkennbar ist. \\Mögliche Ursachen, warum der Effekt nicht so groß, wie erwartet ist, sind: Zum einen ist der Schwierigkeitsgrad der einzelnen Aufgaben sehr unterschiedlich, zum anderen ist es ein Unterschied, ob man aus einer Graphik lediglich Werte ablesen (wie in der Aufgabe Autofahren der Fall) muss, oder die Werte stochastisch (wie in der Aufgabe Zufall der Fall) ausgewertet werden.


Bei der letzten Forschungsfrage (4) wird gefragt, ob bestimmte Aufgabentypen für einzelne Probandengruppen positive oder auch negative Auwirkungen haben. Wie im Ergebnissteil zu sehen ist, werden hierbei drei Fälle einer weiteren Betrachtung unterzogen. 


Die unterdurchschnittliche Bearbeitung \gls{Gr2} bei der Aufgabe \gls{Aut} ($\alpha$ = 0,243 F = 1,407) ist auffällig. Bei dieser Aufgabenstellung war lediglich eine Graphik abgebildet, welche die Geschwindigkeit mit der Zeit der Autofahrerin in Relation setzt. Gefragt war, ob der linke und höhere Teil (also mehr Geschwindigkeit)  des Graphen eine größere Strecke darstellt, wie der gleichbreite (also gleiche Zeitdauer) rechte Teil des Graphen.  Hierbei bedeutet der niedrigere Teil (rechts) eine kürzere zurückgelegte Strecke, da  Weg = Geschwindigkeit mal Zeit gilt. \\
Im Einstiegsproblem, nach der die Gruppen eingeteilt wurden, war in den Graphiken, auf welche sie häufige Fixationen hatten, eindeutig zu sehen, was mit ihnen gemeint war. In der Graphik dieser Aufgabe, kann man die gleichbreiten Graphen (Zeit) auch als Synonym für den zurückgelegten Weg missverstehen. Dieser Effekt tritt jedoch nicht signifikant auf. Graphiken können für die Gruppierung der Problemlöser problematisch wirken.


Weiterhin wurde \gls{Zuf} von der \gls{Gr1} besonders gut gelöst($\alpha$ = 0,152 F = 2,135). In dem Aufgabenteil Zufall sind zwei Graphiken zu sehen, welche beide ausgewertet und verknüpft werden müssen, um die Aufgabenstellung richtig zu beantworten. Der Typ Unsicher hat sich dadurch gebildet, indem er auf der ersten Folie eine besonders lange zweite Runde mit vielen Fixationen auf Bild und Text gezeigt hat. \\
Diese Dauer kann auch als gewissenhaftes Arbeiten einstuft werden, was ihm in der Aufgabe Zufall zum Vorteil wird, da die beiden Graphiken zuerst abgezählt und dann verküpft werden müssen bevor die Aufgabe beantwortet werden kann. Der Unsichere kann als mit einer längeren Bearbeitungszeit ein nicht signifikantes aber durchschnittlich besseres Ergebnis erzielen. 


Im letzten Teil wurde \gls{Ren} von der \gls{Gr3} am besten gelöst ($\alpha$ = 0,092 F = 2,990). Die Aufgabenstellung war: aus einem gegebenen Graphen, welcher die Geschwindlichkeit und die Steckenentfernung in Relation gesetzt hat, den Abstand bis zur längsten geraden Strecke anzugeben. Hierbei konnte dieser Wert einfach aus dem Graphen abgelesen werden, da dieser Wert am Anfang des längsten Teiles ohne Kurve, also Geschwindlichkeitsverlust, liegt. \\
Die Gruppe Textuell hat sich auf der ersten Folie durch viele Fixationen auf den Text auszezeichnet. Da der Abstraktionsgrad der Aufgabe Zufall nicht besonders hoch ist, kann durch gewissenhaftes Lesen der Aufgabestellung (des Textes) und einfaches Ablesen des Graphen diese Aufgabe erfolgsversprechend gelöst werden. 


Abschließend wenden wir uns noch dem Ergebnis aus der Tabelle \grqq dekorative und essenzielle Bildtypen mit Berücksichtigung der Lerntypen \grqq , welche die verschiedenen Lerntypen mit den Bildtypen in Relation setzt, zu. Hierbei fällt auf, dass die Unterschiede zwischen den Gruppen im Teil der dekorativen Bilder sehr gering sind ($\alpha$ = 0,897 F = 0,109). Jedoch hat im Bereich der essenziellen Bilder der Typ Problemlöser schlechter abgeschnitten hat ($\alpha$ = 0,130 F = 2,287).\\
Dieses Ergebnis steht im Einklang mit der Vermutung aus der Diskussion der Forschungsfrage 1. Diese besagt, dass die Problemlöser nicht viele Fixationen auf den Bildern haben, weil sie sie verstanden haben, sondern weil es ihnen schwer fällt sie zu verstehen. Diese Ergebnisse sind jedoch nicht signifikant und geben somit lediglich eine Tendenz an. 

\chapter{Ausblick}
Die Studie zeigt uns, dass sich Lernende anhand von ihren Blickbewegungen in einem gewissen Maße in Gruppen einteilen lassen. Diese Einteilung lässt aber keine direkten Schlüsse auf die mathematische Kompetenz der Probandinnen und Probanden zu.  Im Bereich der Verwendung von unterschiedlichen Darstellungsformen konnten positive Auswirkungen von essenziellen Bildtypen gezeigt werden.  Auf der Grundlage dieser Ergebnisse sollten sich Lehrkräfte stets überlegen, welche Art der Unterstützung sie bei einer Aufgabe anbieten. Dies bietet den Lehrerinnen und Lehrern eine weitere Differenzierungsmöglichkeit zu der in der Einleitung angeschnittenen Prozessdimension (cognitive process Dimension) und der Wissensdimension (Knowledge Dimension), welche von \citeauthor{anderson2001taxonomy} (\citedate{anderson2001taxonomy})l differenziert wurde. Ebenso ist es gut möglich, die \gls{SuS} selbst im Laufe eines Arbeitsauftrages zu einer solchen essenziellen bildlichen Darstellung kommen zu lassen. Dies kann beispielsweise eine Skizze oder ein Graph sein. Aufgabentypen wie gerade beschrieben schaffen eine anschauliche Brücke zwischen der Modellierung und der Lösung des Problems.


Die Auswertung der Studie hat mich zunächst überrascht. Nach der Einteilung der Gruppen bin ich davon ausgegangen, dass die SuS, die sich auf die Abbildungen konzentrieren und somit zur Gruppierung Problemlöser gehören, besser wären. Meine bisherigen Unterrichtserfahrungen legen des Schluss nahe, SuS erhalten am besten einen Zugang zu einem mathematischen Problem, wenn sie eine Anschauung zu dem Thema entwickeln. Wie im Diskussionsteil bereits geschildert, ist diese Erfahrung nicht unbedingt widersprüchlich zu den Ergebissen, die Differenz kann sehr wohl auch der gewählten Gruppeneinteilung geschuldet sein. 


Weitere Studien können diese Daten als Grundlage verwenden, um sich neuen Forschungsfragen zu stellen und hierbei beispielsweise andere Unterteilungen anwenden. Eine Möglichkeit wäre, die Auswertung in umgekehrter Reihenfolge anzugehen: Hierbei werden die Probanden mit guten Leistungen in eine Gruppe zusammengefasst und danach untersucht, ob sie Gemeinsamkeiten in ihren Eyetrackingdaten aufweisen. Somit würde die Gruppeneinteilung nicht aus vorüberlegten Ansätzen entstehen, sondern als Resultat der Auswertung. 
Ebenso wäre es interessant, die gegebene Studie mit einer deutlich größeren Anzahl an Probanden durchzuführen, um deren statistische Aussagekraft zu verbessern.



\printbibliography
\clearpage

\listoffigures % Abbildungsverzeichnis

\printacronyms[title={Abkürzungsverzeichnis}] % Abkürzungsverzeichnis

\listoftables % Tabellenverzeichnis

\onehalfspacing
\end{document}