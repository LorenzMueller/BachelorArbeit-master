\chapter{Einleitung}

In der Ausbildung von \gls{SuS} und Studierenden ist es für jeden Einzelnen unerlässlich, Wege zu finden, sich Lernstoff anzueignen und diesen auch sinnvoll anwenden zu können. Lerntypen können in auditive, visuelle, motorische, und kommunikative unterteilt werden. Der Auditive tut sich beim Lernen über Zuhören am Leichtesten, der Visuelle über Veranschaulichungen, der Motorische eignet sich den Stoff über Bewegungenen an und der Kommunikative nimmt den Lerngegenstand am besten auf, wenn er darüber mit anderen spricht.  Somit sind diese Typen sehr unterschiedlich in ihrer Verwendung von Verarbeitungskanälen (Sinnesorganen).
Jeder dieser Typen schafft es also auf verschiedene Art und Weise, sich neuen Stoff effektiv anzueignen.
Um dies zu gewährleisten, wird vom \cite{LehrplanGrundschule} (Staatsinstitut für Schulqualität und Bildungsforschung) festgehalten:
Es ist "... Aufgabe aller Bildungsorte, in allen Lebensphasen und -bereichen \textbf{individuelles}
Lernen anzuregen und so zu unterstützen, dass es lebenslang selbstverständlich wird."

Hierbei entwickeln sich im Laufe der Schulzeit eigene Strategien, um komplexe Sachverhalte
wie zum Beispiel mathematische Beweise zu erlernen. 
Im Umgang des Lernens aus Texten erscheint es sinnvoll, eine andere Lerntypunterteilung anzustreben, da Worte oder Bilder vorerst gelesen oder betrachtet werden müssen, bevor sie verarbeitet werden können. Somit muss auch ein auditiver oder ein kommunikativer Lerntyp vorerst den Text lesen, um den Inhalt für sich verfügbar zu machen. 

So wie sich die Lernenden in ihren Typen unterscheiden lassen, ist es auch bei den Hilfestellungen zu Aufgaben. Eine Aufgabenstellung kann nicht nur in ihrer kognitiven Prozessdimension (cognitive process dimension) oder ihrer Wissensdimension (knowledge dimension) nach \citeauthor{anderson2001taxonomy} (\citedate{anderson2001taxonomy}), sondern auch in der Art der Unterstützung eingeteilt werden. Im mathematischen Kontext werden diese Unterstützungen meist in Beispielrechnungen oder Graphiken wiedergegeben. Graphiken können laut Zitation in unterschiedliche Kategorien eingeteilt werden, welche für die Bearbeitung der Aufgabestellung unterschiedlich hilfreich sind. 

In vorliegender Studie wird die visuelle Wahrnehmung genauer untersucht und die Effektivität von unterschiedlichen Lerntypen im Lösen von unterschiedlichen mathematischen Aufgabestellungen untersucht. 