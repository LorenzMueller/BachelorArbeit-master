\chapter{Ausblick}
Die Studie zeigt uns, dass sich Lernende anhand von ihren Blickbewegungen in einem gewissen Maße in Gruppen einteilen lassen. Diese Einteilung lässt aber keine direkten Schlüsse auf die mathematische Kompetenz der Probandinnen und Probanden zu.  Im Bereich der Verwendung von unterschiedlichen Darstellungsformen konnten positive Auswirkungen von essenziellen Bildtypen gezeigt werden.  Auf der Grundlage dieser Ergebnisse sollten sich Lehrkräfte stets überlegen, welche Art der Unterstützung sie bei einer Aufgabe anbieten. Dies bietet den Lehrerinnen und Lehrern eine weitere Differenzierungsmöglichkeit zu der in der Einleitung angeschnittenen Prozessdimension (cognitive process Dimension) und der Wissensdimension (Knowledge Dimension), welche von \citeauthor{anderson2001taxonomy} (\citedate{anderson2001taxonomy})l differenziert wurde. Ebenso ist es gut möglich, die \gls{SuS} selbst im Laufe eines Arbeitsauftrages zu einer solchen essenziellen bildlichen Darstellung kommen zu lassen. Dies kann beispielsweise eine Skizze oder ein Graph sein. Aufgabentypen wie gerade beschrieben schaffen eine anschauliche Brücke zwischen der Modellierung und der Lösung des Problems.


Die Auswertung der Studie hat mich zunächst überrascht. Nach der Einteilung der Gruppen bin ich davon ausgegangen, dass die SuS, die sich auf die Abbildungen konzentrieren und somit zur Gruppierung Problemlöser gehören, besser wären. Meine bisherigen Unterrichtserfahrungen legen des Schluss nahe, SuS erhalten am besten einen Zugang zu einem mathematischen Problem, wenn sie eine Anschauung zu dem Thema entwickeln. Wie im Diskussionsteil bereits geschildert, ist diese Erfahrung nicht unbedingt widersprüchlich zu den Ergebissen, die Differenz kann sehr wohl auch der gewählten Gruppeneinteilung geschuldet sein. 


Weitere Studien können diese Daten als Grundlage verwenden, um sich neuen Forschungsfragen zu stellen und hierbei beispielsweise andere Unterteilungen anwenden. Eine Möglichkeit wäre, die Auswertung in umgekehrter Reihenfolge anzugehen: Hierbei werden die Probanden mit guten Leistungen in eine Gruppe zusammengefasst und danach untersucht, ob sie Gemeinsamkeiten in ihren Eyetrackingdaten aufweisen. Somit würde die Gruppeneinteilung nicht aus vorüberlegten Ansätzen entstehen, sondern als Resultat der Auswertung. 
Ebenso wäre es interessant, die gegebene Studie mit einer deutlich größeren Anzahl an Probanden durchzuführen, um deren statistische Aussagekraft zu verbessern.