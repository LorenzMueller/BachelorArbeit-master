\chapter{Diskussion}

Nach der Betrachtung aller Einzelergebnisse können Rückschlüsse auf die Forschungsfragen vollzogen werden.  

Wie aus der Graphik der Gruppenvarianz entnommen werden kann, unterscheiden sich die Gruppen untereinander in folgenden Merkmalen: Fixationen auf den Text bzw. die Bilder und die Dauer des jeweiligen zweiten Durchgangs.

In der Dauer des ersten Durchgangs ist kein signifikanter Unterschied zwischen den Gruppen festzustellen ($\alpha$ = 0,620 F = 0,490). Dieses Ergebnis stimmt mit der Theorie überein, dass die \gls{PuP} den Text zunächst linear lesen und sich erst danach die Unterschiede ergeben. In der Anzahl der Fixationen auf den Text sowie das Bild im zweiten Durchgang ist ein signifikanter Unterschied festzustellen ($\alpha$ = 0,000 F = 11,991 für das Bild und $\alpha$ = 0,001 F = 10,034 für den Text). Ebenso ist in der Dauer des zweiten Durchgangs ein signifikanter Unterschied zu erkennen ($\alpha$ = 0,000 F = 17,073). Bemnach ist die Forschungsfrage (1) mit ja zu beantworten; man kann die \gls{PuP} in Gruppen unterscheiden.

Verwunderlich ist bei der Auswertung der Mittelwerte der Punkte (Forschungsfrage 2), dass die Problemlöser mit 5,5 Punkten im Schnitt am schlechtesten unter den Gruppen abgeschlossen haben. Dieses Ergebnis ist dahingehend verwunderlich, da in der Lerntypeinteilung von Gangé die Problemlöser, welche sich viel mit den unterschiedlichen Darstellungsformen auseinander setzen, hierachisch über den Textuellen, welche sich aus den Grundformen des Lernes nach Gangé zusammen setzen sollten, stehen. \\
Auf der anderen Seite hat Zech seine Lerntypeinteilung nicht mehr hierarchisch vollzogen, was die Ergebnisse weniger verwunderlich erscheinen lässt. Typ Textuell hat in dieser Auswertung mit 6,3 Punkten im Mittel am besten abgeschlossen und der Typ Unsicher liegt mit 6,17 Punkten im Mittelfeld. Die Punktevarianzabbildung zeigt aber auch, dass die Unterschiede bei den Punkten der einzelnen Gruppen nicht signifikant sind ($\alpha$ = 0,551 F = 0,615). 

Das folgende Zitat von Hyrskykari, Ovaska, Majaranta, Räihä und Lehtinen (2008) erklärt anschaulich, wie leicht es ist, Fehlbeurteilungen bei der Untersuchung von Eyetrackingdaten anzustellen:

``Eine markante Zone auf einer Heatmap wird oft als eine interressante Zone interpretiert. Es hat die Aufmerksamkeit des Benutzers auf sich gezogen und deswegen wird angenommen, dass die Zone für den Benutzer jetzt verständlich ist. Dennoch kann aber auch das Gegenteil der Fall sein: Die Zone hat die Aufmerksamkeit des Benutzers angeregt, weil sie verwirrend und problematisch ist und der Benutzer die dargestellte Information nicht verstanden hat''\cite{hyrskykari2008gaze}.

Ebenso haben Hayhoe und Ballard (2005) in ihrer Studie gezeigt, dass in manchen Situationen Benutzer direkt auf ein relevantes und wichtiges Objekt schauen und dennoch keine Gedächtnisspur (Wissensgewinn) regestrierbar ist\cite{hayhoe2005eye}. 

Daraus lässt sich schleißen, dass es nicht so einfach ist, die Verhaltensmuster (in unserer Studie Blickbewegungen) der Probanden direkt in ihre kognitiven Prozesse zu übersetzen. Somit ist es leicht, sich von den Eyetrackingdaten blenden zu lassen und diese falsch zu interpretieren. In dieser Studie könnten die \gls{PuP} aus der Gruppe Problemlöser lediglich viele Fixationen auf den Bildern haben, weil sie diese nicht ganz verstanden hatten. 

Ebenso ist die Gruppengröße von 6 \gls{PuP} in Gruppe Unsicher, 10 Versuchsteilnehmer in Gruppe Problemlöser und 6 \gls{PuP} in Gruppe Textuell deutlich zu gering, um signifikante Ergebnisse zu generieren. Man könnte auch feststellen, diese Gruppengröße bietet zu wenig statistische Aussagekraft für eindeutige Ergebnisse. 

Widmet man sich der dritten Forschungsfrage:\\
Inwiefern haben die Bilder einen Einfluss auf das Ergebnis der \gls{PuP}, lässt sich schon in den ersten beiden Aufgaben einen Wiederspruch feststellen: \gls{Rad} 1 und \gls{Rad} 2 verwenden beide ein dekoratives Bild und ihre Beantwortung fällt im ersten Aufgabenteil sehr gut mit 97,5 Prozent aus und im zweiten Teil, bei dem dasselbe dekorative Bild verwendet wurde sehr schlecht mit 17,9 Prozent aus. Der deutliche Unterschied in der Beantwortung legt den Schluss nahe, das sich das Niveau der Aufgabenstellung unterscheidet.\\
Die Aufgaben \gls{Ber} wurden zu 76,9 und zu 74,3 Prozent richtig bearbeitet, obwohl das beigelegte Bild ebenso nur dekorativ war. In den restlichen Fällen war die verwendete Graphik jeweils essenziell und wurde mit \gls{Aut} 94,9 Prozent, \gls{Zuf} 51,3 Prozent und \gls{Ren} 76,9 Prozent richtig beantwortet. Hierbei wird die Bearbeitung des Aufgabenteils \gls{Zuf} auffällig, da sie am schlechtesten von den Aufgabenstellungen mit essenziellen Graphiken bearbeitet wurde. 


Im Durchschnitt aller \gls{PuP} wurden die Aufgaben mit dekorativen Bildern zu 66,6 Prozent richtig beantwortet und die Aufgabe mit essenziellen Bildern zu 74,4 Prozent. Dies ist zwar kein signifikanter Untersied ($\alpha$ = 0,749 F = 0,115), zeigt aber eine Tendenz in Richtung des Effekts, der aus der Studie von Matthias Böckmann und Stanislaw Schukajlow (2018) erkennbar ist. \\Mögliche Ursachen, warum der Effekt nicht so groß, wie erwartet ist, sind: Zum einen ist der Schwierigkeitsgrad der einzelnen Aufgaben sehr unterschiedlich, zum anderen ist es ein Unterschied, ob man aus einer Graphik lediglich Werte ablesen (wie in der Aufgabe Autofahren der Fall) muss, oder die Werte stochastisch (wie in der Aufgabe Zufall der Fall) ausgewertet werden.


Bei der letzten Forschungsfrage (4) wird gefragt, ob bestimmte Aufgabentypen für einzelne Probandengruppen positive oder auch negative Auwirkungen haben. Wie im Ergebnissteil zu sehen ist, werden hierbei drei Fälle einer weiteren Betrachtung unterzogen. 


Die unterdurchschnittliche Bearbeitung \gls{Gr2} bei der Aufgabe \gls{Aut} ($\alpha$ = 0,243 F = 1,407) ist auffällig. Bei dieser Aufgabenstellung war lediglich eine Graphik abgebildet, welche die Geschwindigkeit mit der Zeit der Autofahrerin in Relation setzt. Gefragt war, ob der linke und höhere Teil (also mehr Geschwindigkeit)  des Graphen eine größere Strecke darstellt, wie der gleichbreite (also gleiche Zeitdauer) rechte Teil des Graphen.  Hierbei bedeutet der niedrigere Teil (rechts) eine kürzere zurückgelegte Strecke, da  Weg = Geschwindigkeit mal Zeit gilt. \\
Im Einstiegsproblem, nach der die Gruppen eingeteilt wurden, war in den Graphiken, auf welche sie häufige Fixationen hatten, eindeutig zu sehen, was mit ihnen gemeint war. In der Graphik dieser Aufgabe, kann man die gleichbreiten Graphen (Zeit) auch als Synonym für den zurückgelegten Weg missverstehen. Dieser Effekt tritt jedoch nicht signifikant auf. Graphiken können für die Gruppierung der Problemlöser problematisch wirken.


Weiterhin wurde \gls{Zuf} von der \gls{Gr1} besonders gut gelöst($\alpha$ = 0,152 F = 2,135). In dem Aufgabenteil Zufall sind zwei Graphiken zu sehen, welche beide ausgewertet und verknüpft werden müssen, um die Aufgabenstellung richtig zu beantworten. Der Typ Unsicher hat sich dadurch gebildet, indem er auf der ersten Folie eine besonders lange zweite Runde mit vielen Fixationen auf Bild und Text gezeigt hat. \\
Diese Dauer kann auch als gewissenhaftes Arbeiten einstuft werden, was ihm in der Aufgabe Zufall zum Vorteil wird, da die beiden Graphiken zuerst abgezählt und dann verküpft werden müssen bevor die Aufgabe beantwortet werden kann. Der Unsichere kann als mit einer längeren Bearbeitungszeit ein nicht signifikantes aber durchschnittlich besseres Ergebnis erzielen. 


Im letzten Teil wurde \gls{Ren} von der \gls{Gr3} am besten gelöst ($\alpha$ = 0,092 F = 2,990). Die Aufgabenstellung war: aus einem gegebenen Graphen, welcher die Geschwindlichkeit und die Steckenentfernung in Relation gesetzt hat, den Abstand bis zur längsten geraden Strecke anzugeben. Hierbei konnte dieser Wert einfach aus dem Graphen abgelesen werden, da dieser Wert am Anfang des längsten Teiles ohne Kurve, also Geschwindlichkeitsverlust, liegt. \\
Die Gruppe Textuell hat sich auf der ersten Folie durch viele Fixationen auf den Text auszezeichnet. Da der Abstraktionsgrad der Aufgabe Zufall nicht besonders hoch ist, kann durch gewissenhaftes Lesen der Aufgabestellung (des Textes) und einfaches Ablesen des Graphen diese Aufgabe erfolgsversprechend gelöst werden. 


Abschließend wenden wir uns noch dem Ergebnis aus der Tabelle \grqq dekorative und essenzielle Bildtypen mit Berücksichtigung der Lerntypen \grqq , welche die verschiedenen Lerntypen mit den Bildtypen in Relation setzt, zu. Hierbei fällt auf, dass die Unterschiede zwischen den Gruppen im Teil der dekorativen Bilder sehr gering sind ($\alpha$ = 0,897 F = 0,109). Jedoch hat im Bereich der essenziellen Bilder der Typ Problemlöser schlechter abgeschnitten hat ($\alpha$ = 0,130 F = 2,287).\\
Dieses Ergebnis steht im Einklang mit der Vermutung aus der Diskussion der Forschungsfrage 1. Diese besagt, dass die Problemlöser nicht viele Fixationen auf den Bildern haben, weil sie sie verstanden haben, sondern weil es ihnen schwer fällt sie zu verstehen. Diese Ergebnisse sind jedoch nicht signifikant und geben somit lediglich eine Tendenz an. 
