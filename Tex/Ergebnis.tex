\chapter{Ergebniss}

Die in der Stuide getesteten Probanden waren Studenten der TU München und wiesen die folgenden Werte auf:

    \begin{itemize}
        \item 35,9 Prozent von ihnen waren weiblich 
        \item die Probanden kamen überwiegend aus Bayern 66,7 Prozent
        \item das Druchschnittsalter lag bei 21,2 Jahren 
        \item waren im Mittel im 3,85. Fachsemester
    \end{itemize}

 Um die Vorschungsfrage (1) zu untersuchen wurden Gruppen mit Hilfe einer Varianzanalyse auf die Werte: Fixationen auf dem Bild, Fixationen auf dem Text und Dauer nach dem ersten Durchgangs, sowie der Dauer des ersten und des zweiten Durchgangs untersucht.


\begin{figure}[!ht]
\noindent\hspace{0.5mm}\includegraphics[width=15cm]{./Ressourcen/Gruppenunterscheidung.png}
\caption{Gruppenvarianz, Lorenz Müller}
\end{figure}

Wenn man sich der zweiten Vorschungsfrage widment. Kommt man zu folgenden Ergebnissen:
Alle Studenten erziehlten im Schnitt 6,03 Punkte. 
Die jeweiligen Gruppen erziehlten insgesamt über alle Aufgaben im Durchschnitt folgende Punkte:

\begin{table}[!h]
\hspace{-5pt}
\begin{tabularx}{\textwidth + 5pt}{| @{\hspace{3pt}} M || @{\hspace{3pt}} M  | @{\hspace{3pt}} M | @{\hspace{3pt}} M |}
\hline
\textbf{ } & \textbf{Typ Unsicher} & \textbf{Typ Problemlöser} & \textbf{Typ Textuell}\\
\hline
\hline
Punkte        & 6,17 & 5,5 & 6,3\\
\hline
\end{tabularx}
\caption{Mittelwert der Punkte}
\end{table}


Bei der Untesuchung nach signifikanten Unterschieden wurde
die Varianzanalyse über die Gesamtpunktzahl angestellt, welche folgende Ergebnisse hervorbringt:

\begin{figure}[!ht]
\noindent\hspace{0.5mm}\includegraphics[width=15cm]{./Ressourcen/Punktevarianz.png}
\caption{Punktevarianz, Lorenz Müller}
\end{figure}

Wendet man sich nun der dritten Forschungsfrage, des Einflusses der gewählten Bilder zu, ergibt sich.
Betrachtet man die Beantwortung der einzelnen Aufgaben von allen Probanden.

\begin{table}[!h]
\hspace{-5pt}
\begin{tabularx}{\textwidth + 5pt}{| @{\hspace{3pt}} M || @{\hspace{3pt}} M  | @{\hspace{3pt}} M | @{\hspace{3pt}} M |}
\hline
\textbf{Aufgabe} & \textbf{Radfahrerin 1} & \textbf{Radfahrerin 2} & \textbf{Rennfahrer} \\
\hline
\hline
Prozent richtig beantwortet       & 97,5 & 17,9 & 76,9 \\
\hline
\end{tabularx}
\caption{Mittelwert der Punkte}
\end{table}

\begin{table}[!h]
\hspace{-5pt}
\begin{tabularx}{\textwidth + 5pt}{| @{\hspace{3pt}} M | @{\hspace{3pt}} M  | @{\hspace{3pt}} M | @{\hspace{3pt}} M |}
\hline
\textbf{Zufall} & \textbf{Berg 1} & \textbf{Berg 2} & \textbf{Autofahren}\\
\hline
\hline
    51,3 & 76,9 & 74,3 &  94,9\\
\hline
\end{tabularx}
\caption{Mittelwert der Punkte}
\end{table}

Im Schnitt wurden somit die jeweiligen Bildtypen zu folgenden Prozentwerten richtig beantwortet. 

\begin{table}[!h]
\hspace{-5pt}
\begin{tabularx}{\textwidth + 5pt}{| @{\hspace{3pt}} M || @{\hspace{3pt}} M  | @{\hspace{3pt}} M | @{\hspace{3pt}} M |}
\hline
\textbf{Aufgabentyp} & \textbf{dekorativ} & \textbf{essenziell} \\
\hline
\hline
Prozent richtig beantwortet       & 66,7 & 74,3 \\
\hline
\end{tabularx}
\caption{Mittelwert der Punkte}
\end{table}

Bei der Untersuchung nach signifikanten Unterschieden ergibt sich bei der Varianzanalyse:


\begin{figure}[!ht]
\noindent\hspace{0.5mm}\includegraphics[width=15cm]{./Ressourcen/Aufgabenuntescheidung.png}
\caption{Punktevarianz, Lorenz Müller}
\end{figure}




Wiedmet man sich nun der vierten Vorschungsfrage:
Werden bei der Auswetung der einzelnen Gruppen auf die Fragen Ergebnisse fett markiert, welche sich um mehr als 20 Prozent vom Wert der Allgemeinheit unterscheiden:

\begin{table}[!h]
\hspace{-5pt}
\begin{tabularx}{\textwidth + 5pt}{| @{\hspace{3pt}} M || @{\hspace{3pt}} M  | @{\hspace{3pt}} M | @{\hspace{3pt}} M |}
\hline
\textbf{Aufgabe} & \textbf{Radfahrerin 1} & \textbf{Radfahrerin 2} & \textbf{Rennfahrer} \\
\hline
\hline
Prozent richtig beantwortet       & 100 & 10 & 90 \\
\hline
\end{tabularx}
\caption{Typ Problemlöser bei den unteschiedlichen Aufgabenstellungen 1}
\end{table}


\begin{table}[!h]
\hspace{-5pt}
\begin{tabularx}{\textwidth + 5pt}{| @{\hspace{3pt}} M | @{\hspace{3pt}} M  | @{\hspace{3pt}} M | @{\hspace{3pt}} M |}
\hline
\textbf{Zufall} & \textbf{Berg 1} & \textbf{Berg 2} & \textbf{Autofahren}\\
\hline
\hline
    40 & 60 & 70 &  \textbf{50}\\
\hline
\end{tabularx}
\caption{Typ Problemlöser bei den unteschiedlichen Aufgabenstellungen 2}
\end{table}

\begin{table}[!h]
\hspace{-5pt}
\begin{tabularx}{\textwidth + 5pt}{| @{\hspace{3pt}} M || @{\hspace{3pt}} M  | @{\hspace{3pt}} M | @{\hspace{3pt}} M |}
\hline
\textbf{Aufgabe} & \textbf{Radfahrerin 1} & \textbf{Radfahrerin 2} & \textbf{Rennfahrer} \\
\hline
\hline
Prozent richtig beantwortet       & 83 & 17 & 67 \\
\hline
\end{tabularx}
\caption{Typ Unsicher bei den unteschiedlichen Aufgabenstellungen 1}
\end{table}

\begin{table}[!h]
\hspace{-5pt}
\begin{tabularx}{\textwidth + 5pt}{| @{\hspace{3pt}} M | @{\hspace{3pt}} M  | @{\hspace{3pt}} M | @{\hspace{3pt}} M |}
\hline
\textbf{Zufall} & \textbf{Berg 1} & \textbf{Berg 2} & \textbf{Autofahren}\\
\hline
\hline
    \textbf{83} & 83 & 67 &  100\\
\hline
\end{tabularx}
\caption{Typ Unsicher bei den unteschiedlichen Aufgabenstellungen 2}
\end{table}

\begin{table}[!h]
\hspace{-5pt}
\begin{tabularx}{\textwidth + 5pt}{| @{\hspace{3pt}} M || @{\hspace{3pt}} M  | @{\hspace{3pt}} M | @{\hspace{3pt}} M |}
\hline
\textbf{Aufgabe} & \textbf{Radfahrerin 1} & \textbf{Radfahrerin 2} & \textbf{Rennfahrer} \\
\hline
\hline
Prozent richtig beantwortet       & 100 & 17 & \textbf{100} \\
\hline
\end{tabularx}
\caption{Typ Textuell bei den unteschiedlichen Aufgabenstellungen 1}
\end{table}


\begin{table}[!h]
\hspace{-5pt}
\begin{tabularx}{\textwidth + 5pt}{| @{\hspace{3pt}} M | @{\hspace{3pt}} M  | @{\hspace{3pt}} M | @{\hspace{3pt}} M |}
\hline
\textbf{Zufall} & \textbf{Berg 1} & \textbf{Berg 2} & \textbf{Autofahren}\\
\hline
\hline
    67 & 67 & 67 &  83\\
\hline
\end{tabularx}
\caption{Typ Textuell bei den unteschiedlichen Aufgabenstellungen 2}
\end{table}

Somit sind folgende Ergebnisse Auffällig. Wobei mit Hilfe der Varianzanalyse untersuch wurde, ob diese sich signifikant von der Allgemeinheit unterscheiden. 

\begin{itemize}
    \item Autofahrt wurde von dem Typ Problemlöser mit nur 50 Prozent richtig gelöst. (Allgemeinheit 94,9 Prozent) ($\alpha$ = 0,243 F = 1,407)
    \item Zufall wurde von dem Typ Unsicher mit 83,3 Prozent richtig gelöst. (Allgemeinheit 0,513) ($\alpha$ = 0,152 F = 2,135)
    \item Rennfahrer wurde von Typ Textuell von allen Probanden richtig gelöst. (Allgemeinheit 0,769) ($\alpha$ = 0,092 F = 2,990)
\end{itemize}

Betrachtet man noch abschliesend, wie sich die unterschiedlichen Lerntypen mit den Bildtypen verhalten haben, so ergibt sich folgende Tabelle.

\begin{table}[!h]
\hspace{-5pt}
\begin{tabularx}{\textwidth + 5pt}{| @{\hspace{3pt}} M | @{\hspace{3pt}} M  | @{\hspace{3pt}} M | @{\hspace{3pt}} M |}
\hline
\textbf{Lerntyp} & \textbf{Unsicher} & \textbf{Problemlöser} & \textbf{Textuell}\\
\hline
\hline
    dekorativ & 62,5 & 60 &  62,5\\
\hline
    essenziell & 83,3 & 60,0 &  83,3\\
\hline
\end{tabularx}
\caption{Typ Textuell bei den unteschiedlichen Aufgabenstellungen 2}
\end{table}

Mit der Varianzanalyse: 

\begin{figure}[!ht]
\noindent\hspace{0.5mm}\includegraphics[width=15cm]{./Ressourcen/DekoEssGruppen.png}
\caption{Punktevarianz, Lorenz Müller}
\end{figure}


